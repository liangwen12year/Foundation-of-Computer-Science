\documentclass{scrartcl}
\usepackage{tikz}
\usetikzlibrary{arrows,automata}
\usepackage{comment}
\usepackage[linguistics]{forest}
\usepackage{textcomp}
\usepackage{makecell}
\usetikzlibrary{automata,positioning}
\usepackage{amsmath,amsfonts,amssymb}
\newcommand{\bl}{{\sqcup}}

\begin{document}
\hfuzz=\maxdimen
\tolerance=10000
\hbadness=10000
\begin{center}
Wen Liang(\texttt{Wen\_Liang@student.uml.edu}) 01724877

\end{center}


\textbf{\underline{Note:The extra question I did was 4.9, 4.13, 4.16, 5.9, 5.22, 5.23}}

\section*{Extra Question:}
\section*{4.9}
Let 
$\sim$ be the binary relation "is the same size". In other words If A 
$\sim$ B, A and B are the same size. We show that 
$\sim$ is an equivalence relation. First, 
$\sim$ is reflexive because the identity function f(x) = x, $\forall$ x $\in$ A is a correspondence f: A $\to$ A. Second, $\sim$ is symmetric because any correspondence has an inverse, which itself is a correspondence. Third, $\sim$ is transitive because if A $\sim$ B via correspendence f, and B $\sim$ C via correspondence g, then A $\sim$ C via correspondence f $\circ$ g (the composition of f and g). Because $\sim$ is reflexive, symmetric, and transitive, $\sim$ is an equivalence relation.

\section*{4.13}
We observe that L(R) $\subseteq$ L(S) if and only if $\overline{L(S)} \cap L(R) = \phi$. The following TM X decides A.\\
X = On input (R,S) where R and S are regular expressions:\\
\quad 1. Construct DFA E such that L(E) = $\overline{L(S)} \cap L(R)$.\\
\quad 2. Run TM T from Theorem 4.4 on input $<E>$, where T decides $E_{DFA}$.\\
\qquad 3. If T accepts, accept. If T rejects, reject.

\section*{4.16}
The folowing TM X decides A.\\
X = On input $<R>$ where R is a regular expression:\\
1. Construct DFA E that accepts $\sum^*$111$\sum^*$.\\
2. Construct DFA B such that L(B) = L(R)$\cap$L(E).\\
3. Run TM T from Theorem 4.4 on input $<B>$,where T decides $E_{DFA}$.\\
4. If T accepts, reject. If T rejects, accept.

\section*{5.9}
Assume T is decidable and let decider R decide T. Reduce from $A_{TM}$ by constructing a
TM S as follows: \\
S: on input $<M,w>$\\
$.   \qquad  $ 1. create a TM Q as follows:\\
$.   \qquad \qquad \qquad$On input x:\\
$.   \qquad \qquad \qquad \qquad \quad $1. if x does not have the form 01 or 10,reject\\
$.   \qquad \qquad \qquad \qquad \quad $2. if x has the form 01, then accept.\\
$.   \qquad \qquad \qquad \qquad \quad $3. else(x has the form 10), Run M on w and accept if M accepts w.\\
$.   \qquad  $2. Run R on $<Q>$\\
$.   \qquad  $3. Accept if R accepts, reject if R rejects.

Because S decides $A_{TM}$, which is known to be undecidable, we then know that T is not
decidable. 

\section*{5.22}
($\Rightarrow$) If A $\leq$ $A_{TM}$, then A is Turing-recognizable because $A_{TM}$ is Turing recognizable.\\
($\Leftarrow$) If A is Turing-recognizable then there exists some TM R that recognizes A. That is, R would receive an input w and accept if w is in A (otherwise R does not accept). To show that A $\leq$ $_m A_{TM}$, we design a TM that does the following: On input w, writes $<R,w>$ on the tape and halts. It is easy to check that $<R,w>$ is in $A_{TM}$ if and only if w is in A. Thus, we get a mapping reduction of A to $A_{TM}$. 

\section*{5.23}
($\Rightarrow$) If A $\leq _m0^*1^*$, then A is decidable because $0^*1^*$ is a decidable language.\\
($\Leftarrow$) If A is decidable, then there exists some TM R that decides A. That is, R would receive an input w and accept if w is in A, reject if w is not in A. To show A $\leq$, reject if w is not in A. To show A $\leq$ $_m0^*1^*$, we design a TM Q that does the following : On input w, runs R on w. If R accepts, outputs 01; otherwise, outputs 10. It is easy to check that:\\
\\
w $\in$ A $\Leftrightarrow$ output of Q $\in$ $0^*1^*$.\\
\\
Thus, we obtain a mapping reduction of A to $0^*1^*$. 
\\
\\
\\
\\
\\
\\
\\
\\
\\
\\
\section*{Required Question:}
\section*{4.2}
Define the language as C = \{$<M,R>$$\mid$ M is a DFA and R is a regular expression with  L(M) = L(R)   \}\\
Recall that the proof of Theorem 4.5 defines a Turing machine F that decides the
language $EQ_{DFA}$ = \{ $<A, B>$ $\mid$ A and B are DFAs and L(A) = L(B) \}. Then the
following Turing machine T decides C:\\ \\
T = On input $<M, R>$, where M is a DFA and R is a regular expression:\\
1. Convert R into a DFA $D_R$ using the algorithm in the textbook.\\
2. Run TM decider F from Theorem 4.5 on input $<M, D_R>$.\\
3. If F accepts, accept. If F rejects, reject.”


\section*{4.3}
Let $ALL_{DFA}$ = $\{<A> \mid$ A is a DFA that recognizes $\sum^* \}$. The following TM L decides $ALL_{DFA}$.\\

L= On input $<A>$ where A is a DFA:\\
1. Construct DFA B that recognizes $\overline{L(A)}$.\\
2. Run TM T from Theorem 4.4 on input $<B>$, where T decides $E_{DFA}$.\\
3. If T accepts, accept. If T rejects, reject.


\section*{4.4}
Let $A\epsilon_{CFG} = \{<G> \mid G$ is a CFG that generates $\epsilon \}$. The following TM V decides A$\epsilon_{CFG}$.\\
V = " On input $<G>$ where G is a CFG:\\
1. Run TM S from Theorem 4.7 on input $<G,\epsilon>$, where S is a decider for $A_{CFG}$\\
2. If S accepts, accept. If S rejects, reject."

\section*{4.6}
\subsection*{(a)}
No, f is not one-to-one because f(1) = f(3).
\subsection*{(b)}
No, f is not onto because there does not exist x $\in$ X such that f(x) = 10.
\subsection*{(c)}
No, f is not correspondence because f is not one-to-one and onto.
\subsection*{(d)}
Yes, g is one-to-one.
\subsection*{(e)}
Yes, g is onto.
\subsection*{(f)}
Yes, g is a correspondence.

\section*{4.7}
Suppose B is countable and a correspondence f: N $\to$ B exists. We construct x in B that is not paired with anything in N. Let x = $x_1x_2$... . Let $x_i = 0$ if $f(i)_i = 1$, and $x_i = 1$ if $f(i)_i = 0$ where $f(i)_i$ is the ith bit of f(i). Therefore, we ensure that x is not f(i) for any i because it differs from f(i) in the ith symbol, and a contradiction occurs.
\section*{4.8}
We demonstrate a one-to-one f: T $\to$ N. Let $f(i,j,k)=2^i3^j5^k$. Function  f is one-to-one because if a $\neq$ b, f(a) $\neq$ f(b). Therefore, T is countable.

\section*{5.1}
Suppose for a contradiction that $EQ_{CFG}$ were decidable. We construct a decider M for $ALL_{CFG} = \{<G> \mid$ G is a CFG and L(G) =$ \sum^*\}$ as follows:\\
\\
M = "On input $<G>$:\\
\qquad 1. Construct a CFG H such that L(H) = $\sum^*$\\
\qquad 2. Run the decider for $EQ_{CFG}$ on $<G,H>$.\\
\qquad 3. If it accepts, accept. If it rejects, reject."\\
\\
M decides $ALL_{CFG}$ assuming a decider for $EQ_{CFG}$ exists. Since we know $ALL_{CFG}$ is undecidable, we have a contradiction.

\section*{5.2}
Here is a Turing Machine M which recognizes the complement of $EQ_{CFG}$:\\
M = "On input $<G,H>$:\\
\qquad 1. Lexicographically generate the strings x $\in$ $\sum^*$.\\
\qquad 2. For each such string x:\\
\qquad 3. \qquad Test whether x $\in$ L(G) and whether x $\in$ L(H), using the algorithm for $A_{CFG}$.\\
\qquad 4. \qquad If one of the tests accepts and the other rejects, accept; otherwise, continue."

\section*{5.3}
[$\frac{ab}{abab}$] [$\frac{ab}{abab}$] [$\frac{aba}{b}$] [$\frac{b}{a}$] [$\frac{b}{a}$] [$\frac{aa}{a}$] [$\frac{aa}{a}$] 

\section*{5.4}
No, For example, define the languages $A = \{0^n1^n \mid n \geq 0 \}$ and B = \{ 1 \}, both over the alphabet $\sum = \{ 0, 1\}$. Define the function f : $\sum^* \to \sum^*$ as\\
\\
\qquad 
\begin{equation}
f(w)=
\begin{cases}
1& \text{if w $\in$ A}\\
0& \text{if w $\notin$ A}
\end{cases}
\end{equation}

Observe that A is a context-free language, so it is also Turing-decidable. Thus, f is a
computable function. Also, w $\in$ A if and only if f(w) = 1, which is true if and only if f(w) $\in$ B. Hence, A $\leq$ $_mB$. Language A is nonregular, but B is regular since it is
finite.

\end{document}
