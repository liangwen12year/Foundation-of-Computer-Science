\documentclass{scrartcl}
\usepackage{tikz}
\usetikzlibrary{arrows,automata}
\usepackage{comment}
\usepackage[linguistics]{forest}
\usepackage{textcomp}
\usepackage{makecell}
\usetikzlibrary{automata,positioning}
\usepackage{amsmath,amsfonts,amssymb}
\newcommand{\bl}{{\sqcup}}

\begin{document}
\hfuzz=\maxdimen
\tolerance=10000
\hbadness=10000
\begin{center}
Wen Liang(\texttt{Wen\_Liang@student.uml.edu}) 01724877

\end{center}
\textbf{\underline{Note:The extra question I did was 6.2, 7.9, 7.10, 7.12}}
\section*{Extra Question:}
\section*{6.2}
Assume for sake of contradiction that there exists some infinite subset of
$MIN_{TM}$ that is Turing-Recognizable. Let this subset be S. Since S is Turing-recognizable, we have that there exists some enumerator E that
enumerates S. Consider the following turing machine M:
\\
M = ”On input x:\\
1. Obtain own source code $<M>$\\
2. Use E to enumerate elements of S until you find a TM T output by E such that
$\mid<M>\mid\quad  < \quad\mid<T>\mid$ \\
3. Output T(x).”\\
\\
We have that step 1 is possible by the recursion theorem. In step 2 will eventually
find such a TM T because S is infinite and hence its elements’ lengths are not bounded.
Hence, we have that the above Turing machine M effectively simulates T on all it’s
inputs. i.e., M accepts, rejects and loops on exactly those inputs that are accepted,
rejected and looped on respectively by T. Therefore M is equivalent to T.  But we have
that $\mid<M>\mid\quad < \quad\mid<T>\mid$. This is a contradiction as T is supposed to be minimal. Hence, $MIN_{TM}$ has no infinite subset that is Turing recognizable.


\section*{7.9}

We construct a TM M that decides $TRIANGLE$ in polynomial time.\\
M = " On input $<G>$ where G is a graph:\\
1. For each triple of vertices $v_1$, $v_2$, $v_3$ in G:\\
2. \qquad If edges $(v_1, v_2), (v_1, v_3)$, and $(v_2, v_3)$, are all edges of G,accept.\\
3. No triangle has been found in G, so reject."\\
\\
A graph with m vertices has $m!/(3!(m-3)!) = O(m^3)$ triples of vertices. Therefore, stage 2 will be repeated at most $O(m^3)$ times. In addition, each stage can be implemented to run in polynomial time. Therefore, $TRIANGLE \in P$.

\section*{7.10}
$ALL_{DFA}$ is the set of every language that is deterministically computed. In other words, there is always a finite number of steps to be computed and the machine never has to make any decisions, it always knows its next state. The class P is defined as a class of languages that are decidable in polynomial time on a deterministic single tape Turing machine. So, given a TM where we run an input string w, if it is accepted as decidable in polynomial time, then it is Deterministic and therefore in Class P.
\\
\\
M = "On input$<B,w>$, where B is a DFA and w is a string:\\
1. Simulate B on input w.\\
2. If the simulation ends in an accept state,accept. If it ends in a non-accepting state, reject."\\
\\
If input is accepted, then it is in class P.



\section*{7.12}
A nondeterministic polynomial time algorithm for $ISO$ operates as follows:\\
"On input $<G,H>$ where G and H are undirected graphs:\\
1. Let m be the number of nodes of G and H. If they do not have the same number of nodes, reject.\\
2. Nondeterministically select a permulation $\pi$ of m elements.\\
3. For each pair of nodes x and y of G check that (x,y) is an edge of G iff $(\pi(x),\pi(y))$ is an edge of H. If all agree, accept. If any differ, reject."\\
\\
Stage 2 can be implemented in polynomial time nondeterministically. Stage 3 takes polynomial time. Therefore $ISO \in $NP.
\\
\\
\\
\\
\section*{Required Question:}

\section*{6.1}
LISP:\\
((LAMBDA (X) (LIST X (LIST (QUOTE QUOTE) X)))\\
(QUOTE (LAMBDA (X) (LIST X (LIST (QUOTE QUOTE) X)))))
\\
\\
Python:\\
a = "print 'a = ', repr(a), '$\backslash$n', repr(a)[1:-5]', a"\\print 'a = ', repr(a),'$\backslash$n', repr(a)[1:-5] \\
\\
C language:\\
\#include $<$stdio.h$>$\\
int main() \{ char *s = "\#include $<$stdio.h$>$\%cint main() \{ char *s = \%c\%s\%c; printf( s, 10, 34, s, 34); return 0; \}"; printf( s, 10, 34, s, 34); return 0;\}




\section*{6.6}
To solve this problem, we have to be careful about circular definitions, that is, defining M in terms of N, and N in terms of M. Remember that q(w) is a T.M. that prints w on its tape, and halts.\\
\\
M = "On input w:\\
\\
(a) Ignore input.\\
(b) Obtain $<M>$ by the recursion theorem.\\
(c) Compute $q(<M>)$, and write it to the tape.\\
(d) Halt"\\
\\
We let N = $q(<M>)$. When M runs, it writes N = $<q(<M>)>$ in its tape, and when N runs, it writes $<M>$ in its tape.  

\section*{7.6}
P is closed under union. For any two P-languages $L_1$ and $L_2$, let $M_1$ and $M_2$ be the TMs that decide them in polynomial time. We construct a TM $M^/$ that decides the union of $L_1$ and $L_2$ in polynomial time:\\
$M^/$="On input $<w>$:\\
1. Run $M_1$ on w. If it accepts, accept.\\
2. Run $M_2$ on w. If it accepts, accept. Otherwise, reject."\\
$M^/$ accepts w if and only if either $M_1$ and $M_2$ accept w. Therefore, $M^/$
decides the union of $L_1$ and $L_2$. Since both stages take polynomial time, the algorithm runs in polynomial time.\\
\\
\\
P is closed under concatenation. For any two P-language $L_1$ and $L_2$, let $M_1$ and $M_2$ be the TMs that decide them in polynomial time. We construct a TM $M^/$ that decides the concatenation of $L_1$ and $L_2$ in polynomial time:\\
$M^/$=" On input $<w>$:\\
1. For each way to cut w into two substrings w = $w_1w_2$:\\
2. Run $M_1$ on $w_1$ and $M_2$ on $w_2$. If both accept, accept.\\
3. If w is not accepted after trying all the possible cuts, reject."\\
$M^/$ accepts w if and only if w can be written as $w_1w_2$ such that $M_1$ accepts $w_1$ and $M_2$ accepts $w_2$. Therefore, $M^/$ decides the concatenation of $L_1$ and $L_2$. Since stage 2 runs in polynomial time and is repeated at most O(n) times, the algorithm runs in polynomial time.\\
\\
\\
P is closed under complement. For any P-language L, let M be the TM that decides it in polynomial time. We construct a TM $M^/$ that decides the complement of L in polynomial time:\\
$M^/$ = " On input $<w>$:\\
1. Run M on w.\\
2. If M accepts, reject. If it rejects, accept."\\
$M^/$ decides the complement of L. Since M runs in polynomial time, $M^/$ also runs in polynomial time.


\section*{7.7}
NP is closed under union. For any two NP-languages $L_1$ and $L_2$, let $M_1$ and $M_2$ be the NTMs that decide them in polynomial time. We construct a NTM $M^/$ that decides the union of $L_1$ and $L_2$ in polynomial time:\\
$M^/$ = " On input $<w>$:\\
1. Run $M_1$ on w. If it accepts, accept.\\
2. Run $M_2$ on w. If it accepts, accept. Otherwise, reject."\\
In both stages 1 and 2. $M^/$ uses its nondeterminism when the machines being run make nondeterministic steps. $M^/$ accepts w if and only if either $M_1$ and $M_2$ accept w.
Therefore, $M^/$ decides the union of $L_1$ and $L_2$. Since both stages take polynomial time, the algorithm runs in polynomial time.
\\
\\
\\
NP is closed under concatenation. For any two NP-languages $L_1$ and $L_2$, let $M_1$ and $M_2$ be the NTMs that decide them in polynomial time. We construct a NTM $M^/$ that decides the concatenation of $L_1$ and $L_2$ in polynomial time:\\
$M^/$ = " On input$<w>$:\\
1. For each way to cut w into two substrings w = $w_1w_2$:\\
2.\qquad Run $M_1$ on $w_1$.\\
3.\qquad Run $M_2$ on $w_2$. If both accept, accept; otherwise continue with the next choice of $w_1$ and $w_2$.\\
4. if w is not accepted after trying all the possible cuts, rejects." \\
In both stages 2 and 3, $M^/$ uses its nondeterminism when the machines being run make nondeterministic steps. $M^/$ accepts w if and only if w can be expressed as $w_1w_2$ such that $M_1$ accepts $w_1$ and $M_2$ accepts $w_2$. Therefore, $M^/$ decides the concatenation of $L_1$ and $L_2$. Since stage 2 , stage 3 runs in polynomial time and is repeated for at most O(n) time, the algorithm runs in polynomial time.

\section*{7.8}
Here is the high-level description of TM M that decides $CONNECTED$ \\
M = " On input $<G>$, the encoding of a graph G:\\
1. Select the first node of G and mark it.\\
2. Repeat the following stage until no new nodes are marked:\\
3.\qquad For each node in G, mark it if it is attatched by an edge to a node that is already marked.\\
4. Scan all the nodes of G to determine whether they are all marked. If they are, accept; otherwise, reject."
\\
\\
Here we give a high-level analysis of the algorithm. This analysis is sufficient to allow us to conclude that it runs in polynomial time. A more detailed analysis would go into the head motion of the TM, but we do not do that here. The algorithm runs in $O(n^3)$ time.\\
Stage 1 takes at most $O(n)$ steps to locate and mark the start node. Stage 2 causes at most n+1 repititions, because each repetition except for the last repetition marks at least one additional node. Each execution of stage 3 uses at most $O(n^2)$ steps by examining all adjacent nodes to see whether any have been marked. Therefore in total, stage 2 and 3 take $O(n^3)$ time. Stage 4 uses $O(n)$ steps to scan all nodes.\\
Therefore the algorithm runs in $O(n^3)$ time and $CONNECTED$ is in P.



\end{document}
